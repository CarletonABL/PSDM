\documentclass[12pt]{article}

\usepackage{amsmath}
\usepackage{pifont}
\usepackage{natbib}
\usepackage{geometry}
\usepackage{fleqn}
\usepackage{txfonts}
\usepackage{hyperref}
\usepackage{graphicx}
\usepackage[procnames]{listings}
\usepackage{color}
\usepackage{placeins}
\usepackage{booktabs}
\usepackage{multirow}
\usepackage{array}
\usepackage{soul}
%\usepackage{multicol}
\usepackage{cleveref}
\usepackage{fancyvrb}
\usepackage{tikz}
\usetikzlibrary{shapes,arrows}
\graphicspath{ {img/} }
\usepackage{enumitem}
\usepackage{algorithm}
\usepackage[noend]{algpseudocode}

\pdfminorversion=6
\newcommand{\btheta}{\bm{\uptheta}}

% Spaces% Math shortcuts
\renewcommand{\b}[1]{\mathbf{#1}}
\newcommand{\bg}[1]{\bm{#1}}
\newcommand{\R}{\mathbb{R}}
\newcommand{\transpose}{\mathsf{T}}
\newcommand{\grad}[1]{{\nabla}#1}
\newcommand{\norm}[1]{\left\lVert#1\right\rVert}
\newcommand{\sign}{\operatorname{sign}}
\newcommand{\diag}[1]{\operatorname{diag}\left[#1\right]}
\renewcommand{\frame}[3]{\,^{#2\hspace{-0.05em}}#1_{#3}}
\def\multiset#1#2{\ensuremath{\left(\kern-.3em\left(\genfrac{}{}{0pt}{}{#1}{#2}\right)\kern-.3em\right)}}

\setcounter{MaxMatrixCols}{20}

% Common symbol shortcuts
\newcommand{\bx}{\b{x}}
\newcommand{\q}{\b{q}}
\newcommand{\qd}{\b{\dot{q}}}
\newcommand{\qdd}{\b{\ddot{q}}}
\newcommand{\btau}{\bm{\uptau}}
\newcommand{\bD}{\b{D}}
\newcommand{\bDq}{\b{D}(\q)}
\newcommand{\bC}{\b{C}}
\newcommand{\bCq}{\b{C}(\q, \qd)}
\newcommand{\bG}{\b{G}}
\newcommand{\bGq}{\bG(\q)}
\newcommand{\bTheta}{\b{\Theta}}
\newcommand{\bY}{\b{Y}}
\newcommand{\bYq}{\bY(\q, \qd, \qdd)}
\newcommand{\bX}{\b{X}}
\newcommand{\bXi}{\bX_i}
\newcommand{\bE}{\b{E}}
\newcommand{\bP}{\b{P}}
\newcommand{\bB}{\b{B}}
\newcommand{\bPhi}{\b{\Phi}}
\newcommand{\br}{\b{r}}
\newcommand{\bJ}{\b{J}}
\newcommand{\bJq}{\b{J}(\q)}
\newcommand{\bJv}{\b{J_v}}
\newcommand{\bJvq}{\b{J_v(\q)}}
\newcommand{\bJw}{\b{J_\omega}}
\newcommand{\bJwq}{\b{J_\upomega(\q)}}
\newcommand{\bT}{\b{T}}
\newcommand{\T}[2]{\frame{\bT}{#1}{#2}}
\newcommand{\bR}{\b{R}}
\newcommand{\bz}{\b{z}}
\newcommand{\bI}{\b{I}}
\renewcommand{\bP}{\b{P}}
\newcommand{\bgamma}{\bm{\upgamma}}
\newcommand{\nminus}{\hspace{-0.75em}}
\newcommand{\wrench}[1]{\mathbbm{#1}}

\usepackage[framed,numbered,autolinebreaks,useliterate]{mcode}

%opening
\title{README -- PSDM Matlab Code Package}
\author{Steffan Lloyd (steffan.lloyd@carleton.ca)}

\begin{document}

\maketitle

PSDM is a numerical method of deriving the equations of motion of an arbitrary rigid body chain, in regressor form. It is an alternative means of deriving the equations of motion for a rigid, serial kinematic chain. The result is a numerically represented model in a highly organized form, which allows for many symbolic manipulations through numerical methods. This allows for

\begin{enumerate}[itemsep=0pt]
 \item The generation of very fast real time code.
 \item Automatic model simplification (to achieve faster evaluation times)
 \item Both forward and inverse dynamic modelling in a single derivation and with the same inertial parameter set.
\end{enumerate}

To use PSDM, you need to first derive a PSDM model. This will return an \textit{exponent matrix} $\mathbf{E}$ and a \textit{reduction page matrix} $\mathbf{P}$. These matrices contain the full information to compute the forward and inverse dynamic model, as
\begin{align}\label{eq:ID}
	\tau_i = \mathbf{y}_p(\q, \qd, \qdd) \, \bP_i \, \mathbf{\Theta}_b,
\end{align}
and $\b{y}_p(\q, \qd, \qdd)$ can be calculated from $\bE$ as
\begin{align}
		y_j(\q, \qd, \qdd) = \prod_{i=1}^{5n} \Big[\gamma_i(\q, \qd, \qdd)\Big]^{e_{i,j}}, \quad e_{i,j} \in \big\{0, 1, 2\big\},\\
		\b{\gamma}(\q, \qd, \qdd) = \begin{bmatrix}\q^\transpose & \sin(\q)^\transpose & \cos(\q)^\transpose & \qd^\transpose & \qdd ^\transpose \end{bmatrix}^\transpose,
\end{align}

\section{Derivation}

Derivation of the PSDM model can be accomplished in a single line of code with one of two calling syntaxes. 

\begin{lstlisting}
% Can derive model from a DH table and a gravity vector.
[E, P] = PSDM.deriveModel(DH, g);

% Or, can instead give an anonymous function which evaluates the 
% inverse dynamics of the manipulator.
[E, P] = PSDM.deriveModel(@inverseDynamicsFunc, jointTypes);
\end{lstlisting}

\subsection{Kinematic Calling Syntax}
If the DH / g syntax is used, the DH table should be given as a $n\times 6$ matrix:
\begin{align}
	\text{DH} = \begin{bmatrix}
		a_1 & \alpha_1 & d_1 & \theta_1 & t_1 & s_1\\
		\vdots &\vdots &\vdots &\vdots &\vdots &\vdots\\
		a_n & \alpha_n & d_n & \theta_n & t_n & s_n
	\end{bmatrix}
\end{align}
where the DH variables are as defined by \cite{spong2008}:
\begin{enumerate}[itemsep=0pt]
	\item $a_i$ is the distance along the x axis from frame $o_{i-1}$ to $o_i$;
	\item $\alpha_i$ is the angular rotation about $x_i$ from $o_{i-1}$ to $o_i$;
	\item $d_i$ is the linear displacement along the $z$-axis from $o_{i-1}$ to $o_i$;
	\item $\theta_i$ the angular rotation about the $z$-axis from $o_{i-1}$ to $o_i$;
	\item $t_i$ is a number representing the joint type -- 0 indicates a revolute joint, 1 indicates a prismatic joint
	\item $s_i$ is either -1 or 1, denoting the direction of the joint.
\end{enumerate}
The variables $t_i$ and $s_i$ are combined such that, for each joint, we have
\begin{align}
	d_i^* &= d_i + t_i s_i q_i &
	\theta_i^* &= \theta_i + (1-t_i)s_i q_i
\end{align}
The gravity vector is a unit vector which points "upwards" (i.e. against gravity). If omitted, a gravity vector of
\begin{align}
	\mathbf{g} = \begin{bmatrix} 0 & 0 & 1\end{bmatrix}^T
\end{align}
is assumed, i.e. that the $z$-axis points directly upwards against gravity.

\subsection{Implicit model simplifications}
A third parameter $\mathbf{X}$ can also be supplied, which represents the inertial parameters of the robot, in the following form:
\begin{align}
	\b{X} = \begin{bmatrix}
		m_1 & r_{x,1} & r_{y,1} & r_{z,1} & I_{xx,1} & I_{yy,1} & I_{zz,1} & I_{xy,1} & I_{xz,1} & I_{yz,1}\\
		\vdots &\vdots &\vdots &\vdots &\vdots &\vdots&\vdots&\vdots&\vdots&\vdots\\
		m_n & r_{x,n} & r_{y,n} & r_{z,n} & I_{xx,n} & I_{yy,n} & I_{zz,n} & I_{xy,n} & I_{xz,n} & I_{yz,n}\\
	\end{bmatrix}
\end{align}
where
\begin{itemize}[itemsep=0pt]
	\item $m_i$ is the mass of link $i$;
	\item $r_{x,i}$, $r_{y,i}$, $r_{z,i}$ are the (x,y,z) position of the center of gravity of the link;
	\item $I_{xx,i}$, $I_{yy,i}$, $I_{zz,i}$, $I_{xy,i}$, $I_{xz,i}$, $I_{yz,i}$ are the principle and cross inertia terms of the link, centered at the center of gravity and aligned with the main coordinate system of the link.
\end{itemize}
Note, if supplied, the numerical values of this variable are not actually used. However, any parameter which is given as \textit{identically} zero will be implicitely numerically ignored by the algorithm, resulting in a simpler equation. If omitted, the algorithm will simply assume all parameters are nonzero.

\subsection{Function calling syntax}

If you do not have a DH table of the robot, but instead have access to some dynamic simulation of your manipulator, you can still use PSDM if you can define an anonymous function `@inverseDynamicsFunc` which allows the algorithm to sample the joint torques for any joint states and inertial parameters, as
\begin{lstlisting}
tau = inverseDynamicsFunc(Q, Qd, Qdd, X, g_scale)
\end{lstlisting}\vspace{1em}
where \texttt{tau}, \texttt{Q}, \texttt{Qd}, \texttt{Qdd} are $n \times N$ matrices ($n$ degrees of freedom, $N$ samples), and $\mathbf{X}$ is an inertia parameter matrix as defined above. Finally, the parameter \texttt{g\_scale} should be accepted as a scalar which scales gravity. For example, if given \texttt{g\_scale = 0}, the model should run as if there were no gravity, and if \texttt{g\_scale = 1}, then the model should run normally (with gravity).

\subsection{Additional Options}

Additionally, the following name/value pairs can be supplied:
\begin{itemize}[itemsep=0pt]
	\item \texttt{tolerance}: A tolerance used in the ID. Any factors below this tolerance (in estimation of the torques) is ignored. Defaut: \texttt{1e-10}.
	\item \texttt{verbose}: A verbosity flag. Set to false to suppress output. Default: \texttt{true}.
	\item \texttt{gravity\_only}: If true, the algorithm will ignore all acceleration  and velocity effects. Default: \texttt{false}.
\end{itemize}

\section{Using the PSDM Model}

\subsection{Determining the Parameter Vector}

Once the PSDM model has been derived, it can be used as follows. First, the minimal parameter vector $\b{\Theta}_b$ must be determined. A regression of \cref{eq:ID} can be used for this, using either numerical or experimental data. To determine $\b{\Theta}_b$ using numerical data, the following function can be used as
\begin{lstlisting}
% Using the DH and g syntax
Theta = PSDM.X2Theta(DH, g, X, opt)
% Or using the functional syntax
Theta = PSDM.X2Theta(inverseDynamicsFunc, linkTypes, X, opt)
\end{lstlisting}

\subsection{Forward and Inverse Dynamics}

With a parameter vector defined, the forward and inverse dynamics can be evaluated with the functions
\begin{lstlisting}
tau = PSDM.inverseDyanmics(E, P, Theta, Q, Qd, Qdd)
Qdd = PSDM.forwardDynamics(E, P, Theta, Q, Qd, tau)
\end{lstlisting}

\subsection{Fast Real-Time Code Generation}
The functions \texttt{PSDM.inverseDynamics} and \texttt{PSDM.forwardDynamics} work fine, but are slow. Dedicated fast functions can be automatically generated from a model as
\begin{lstlisting}
PSDM.makeInverseDynamics(filename, E, P, Theta)
PSDM.makeForwardDynamics(filename, E, P, Theta)
\end{lstlisting}\vspace{1em}
This will generate a matlab function defined as per the name given in the \texttt{filename} variable, with many optimizations done. This code can also be compiled into C code by giving additional options to the above functions. The generated functions are very fast.

\subsection{Complexity Reduction}
This codebase also includes tools to reduce the complexity of the model without significantly reducing model quality. The usage of this is summarized in the help text of the function
\begin{lstlisting}
[Eh, Ph] = PSDM.reduceModelComplexity(DH, g, X, E, P, options)
% Or, using the function syntax
[Eh, Ph] = PSDM.reduceModelComplexity(inverseDynamicsFunc, linkTypes, E, P, options)
\end{lstlisting}

\section{Making PSDM Code}

Many of the functions in PSDM run fairly slow in interpreted Matlab code. These can be sped up by compiling them into MEX files, if the user has Matlab Coder installed.

To do this, run the command:
\begin{lstlisting}
PSDM.make();
\end{lstlisting}

This will compile many of the PSDM functions into mex files. Then, to tell the package to ``use'' the mex files, you need to modify the file \texttt{+PSDM/config.m} such that the \texttt{use\_mex} parameter is \texttt{true}.

\section{Examples}

Several examples are given for common manipulators, including
\begin{enumerate}[itemsep=0pt]
	\item Two-link planar manipulator;
	\item SCARA robot;
	\item KUKA KR6 spherical wrist 6-DOF manipulator;
	\item KUKA LBR7 7-DOF robot.
\end{enumerate}\vspace{1em}
Photos and numbers are given in the appropriately named scripts in the \texttt{examples} folder.

\section{Crediting and Contact}

This software is developped by Steffan Lloyd (email steffan.lloyd@carleton.ca). The code is offered ``as is'' and can be used at the risk of the user.

If you use our work in a project, we ask that you credit us appropriately (through a citation or otherwise). Citation information will be posted once our manuscript on the subject is published.

\bibliographystyle{sageh}
\bibliography{library.bib}

\end{document}